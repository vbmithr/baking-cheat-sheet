\documentclass[a4paper,twocolumn,10pt]{article}
\usepackage{xcolor}
\usepackage{listings}
\usepackage{hyperref}
\usepackage{fullpage}

\lstdefinestyle{BashInputStyle}{
  language=bash,
  basicstyle=\small\sffamily,
  numbers=left,
  numberstyle=\tiny,
  numbersep=3pt,
  frame=tb,
  columns=fullflexible,
  backgroundcolor=\color{yellow!20},
  linewidth=0.9\linewidth,
  xleftmargin=0.1\linewidth
}

\title{Tezos Baking, Theory \& Practice}
\date{\today}
\author{
  Vincent Bernardoff\\
  Tezos Southeast Asia
}

\begin{document}
\twocolumn
\maketitle
\tableofcontents
\section{Introduction}

This document intends to be a handy cheat sheet for Tezos
operators. It details the installation and operation of Tezos
software. It is a complement to the original
documentation\footnote{\url{http://tezos.gitlab.io}} already
available. It summarizes the different steps needed to run a
\emph{node} and associated \emph{baker}, \emph{endorser}, and
optionally \emph{accuser} daemons which are necessary to \emph{bake},
i.e. participate in the Tezos consensus algorithm.

The emphasized terms (in \emph{italic}) or “quoted” is technical
vocabulary that is used in Tezos source code and thus often commonly
used to talk about Tezos concepts. We encourage you to use it when you
discuss Tezos things with your peers since it is more precise and will
better describe what you are talking about.

Tezos uses LPoS (\emph{liquid proof of stake}) to achieve
consensus. The proof of stake algorithm\footnote{Called \emph{Emmy} in
  reference to Emmy Noether} implemented in Tezos nodes attributes
block creation rights (“slots”) to stakeholders at random,
proportional to the size of their stake.

Operators (“bakers”) will need to register as \emph{delegates}, then
the proof of stake algorithm run by nodes will, after some time,
attribute them block creation (“baking”) rights to them provided their
wallets contains at least one \emph{roll}\footnote{Currently, XTZ
  8000} of XTZ.

The proof of stake algorithm also comprise an additional mechanism
which increase
security\footnote{\url{https://blog.nomadic-labs.com/analysis-of-emmy.html}}:
32 \emph{endorsing} rights per block are attributed proportionally to
bakers. Like baking blocks, endorsing blocks is rewarded by the
network hence the bakers are incentivized to do it.

Finally, bakers can also run an \emph{accuser} daemon, which watches
the chain and try to detect reprehensible behaviour of other network
participants. Upon finding such transgressions, the daemon can inject
a \emph{denunciation} operation in the chain, which is also
rewarded. Such transgressions are \emph{double baking} and
\emph{double endorsing}. They respectively mean that a baker baked two
blocks of the same height, or endorsed two different blocks at the
same height. This can be either a deliberate attack on the network or
the result of negligence in running a baking
infrastructure. Historically, only the latter has happened. We will
see later how it happens and how to prevent it.

It is your responsibility as an operator (“baker”) to properly
register as a \emph{delegate}, then launch and ensure proper operation
of at least the \emph{node}, the \emph{baker} and \emph{endorser}
daemons, and optionally the \emph{accuser}.

The following is a brief description of the different components
software and the interaction between those components. Configuration
details and precise instructions on how to run it will be explained in
a later section.

\subsection{\texttt{tezos-node}}

The Tezos node is the cornerstone of the network, the P2P node which
interacts (“gossip”) with the rest of the network and accepts clients
commands (RPCs). It implements \emph{Emmy}, the consensus algorithm,
the \emph{Michelson}\footnote{Tezos' native smart contract language}
virtual machine, and so on. It accepts messages from its P2P peers in
a dedicated (configurable) TCP port and from this information, will
decides which messages it considers genuine and will build a
blockchain from them. It also listens on client requests on another
TCP port (RPC).

The node has multiple modes of operation\footnote{See
  \url{http://tezos.gitlab.io/mainnet/user/history_modes.html}}:

\begin{description}
  \item[archive] The node will maintain a full \emph{context}, i.e. be
    able to answer queries regarding any block on the network, such
    has “what was the balance of address \emph{A} at block \emph{B}
    for any \emph{B}”. Takes a lot of space (about 170 GiB).
   \item[full] The node will maintain a context up to 5 cycles in the
     past. Takes much less space (about 50 GiB). This is the
     \textbf{recommended} mode for baking.
    \item[rolling] This is the lightest mode as it only maintains a
      minimal rolling fragment of the chain data so the node can still
      validate new blocks and synchronize with the
      head. \textbf{Unsuitable for baking}.
\end{description}

All other executables provided with the node are
clients, which role is detailed below.

\subsection{\texttt{tezos-client}}

\emph{tezos-client} is a tool for querying the node and creating
(\emph{injecting}) \emph{operations}, such as transactions, smart
contract origination, voting, and so on. It is not directly related to
baking but is necessary to create the cryptographic keys used by
baking, endorsing and accusing daemons.

\subsection{\texttt{tezos-baker-*}}

The role of \texttt{tezos-baker-*}\footnote{\texttt{*} relates to the
  protocol version, currently \texttt{004-Pt24m4xi}} is to
automatically create block and inject them in the node, which will
broadcast them to its peers. Its operation is as follows:

\begin{footnotesize}
\begin{enumerate}
\item Baker has access to a wallet with delegates
\item Node notifies the baker of new blocks on the chain
\item Baker checks with the shell if its delegates have slots
\item Baker sleeps until its time to bake, then wakes up and…
\item Queries the shell’s current mempool
\item Build a block with operations
\item Inject blocks to the node
\end{enumerate}
\end{footnotesize}

\subsection{\texttt{tezos-endorser-*}}

The role of \texttt{tezos-endorser-*} is to automatically inject
endorsement operations in the node. Its operation is as follows:

\begin{footnotesize}
\begin{enumerate}
\item Endorser has access to a wallet with delegates
\item Shell notifies the endorser of new blocks on the chain
\item Endorser checks with the shell if its delegates have endorsements slots
\item Endorse sleeps until its time to endorse, then wakes up and…
\item Prepare an endorsement operation
\item Inject the endorsement to the node
\end{enumerate}
\end{footnotesize}

\subsection{\texttt{tezos-accuser-*}}

The role of \texttt{tezos-accuser-*} is to automatically inject proof
of other bakers' misbehaviour in the node. Its operation is as
follows:

\begin{footnotesize}
\begin{enumerate}
\item Inspect the injected blocks and endorsement (via the shell)
\item When a delegate behaves in a malicious way:
  \begin{itemize}
    \item A delegate injected two blocks at a same level on two different branches
    \item A delegate endorsed two different blocks at a same level
  \end{itemize}
  → Inject a denunciation operation
\item The offending user will be slashed of all its rewards and bond
  deposits
\end{enumerate}
\end{footnotesize}

\subsection{\texttt{tezos-signer}}

\texttt{tezos-signer} is an optional mechanism useful to provide
additional security in some configurations. The aforementioned daemons
need to access a wallet in order to function. This wallet can either
have local keys (encrypted on disk, or on HSM\footnote{Ledger Nano S
  is supported by default.}). In some situation you would like to put
those keys in a different machine than the one the daemons are running
on. In this situation, you can add \emph{remote keys} to the
client. When used to sign data, the data to be signed will be sent to
a remote server who handles the signing and returns the
result. \texttt{tezos-signer} is such a server compatible with the
rest of Tezos software for remote signing. The configuration and use
of such a server is outside the scope of this document\footnote{More
  info at
  \url{http://tezos.gitlab.io/mainnet/user/key-management.html\#signer}}.

\section{Prerequisites}

Detailed\footnote{\url{http://tezos.gitlab.io/mainnet/introduction/howtoget.html}}
instructions of how to obtain Tezos software can be found of the main
Tezos documentation, and are outside the scope of this document. There
are two ways, either build the software directly from source code
(needs a working OCaml installation with OPAM), or using a Docker
image. Use whichever you feel most comfortable with. OCaml programmers
will probably choose the former, while devops might opt for the
latter.

It is \textbf{mandatory} to use NTP or a similar mechanism to keep
your server on time when running Tezos daemons. Indeed, the consensus
algorithm is synchronous\footnote{As Tezos evolves, this can change in
  the future}, and therefore requires correct system time to function.

Using SSDs is also strongly recommended. Tezos software might improve
its performance in the future and work fine on legacy hard disks.


\section{Prepare}
\subsection{Key management}

Tezos baking requires the utmost care in how cryptographic keys are
handled. Indeed, in Tezos, the key used to sign blocks and
endorsements is the key of your delegate account, which,
currently\footnote{\today} \textbf{must} be an implicit account (tz1
address). This means that you must use a “hot wallet” in order to
sign, and that this wallet must contain enough funds in order to pay
the bonds that are automatically frozen from your wallet each time you
bake or endorse. They will be returned, alongside with rewards, after
5 cycles.

All this implies that bakers are target for hacks, like the hot wallet
of an exchange. There are multiple ways to design a secure baking
infrastructure, and this is mostly outside the scope of this
tutorial. We succinctly present two possibilities, one using encrypted private
keys on disk, and the other using the Ledger Nano S, compatible out of
the box with Tezos software.

\subsubsection{Encrypted on disk}

Tezos allows you to create secret keys that will be stored encrypted
on disk. Upon creating such keys, or using them, a password will be
requested. This means that launching baking daemons will require human
intervention because the password will be requested at startup. Use this:

\lstinline[style=BashInputStyle]'$ tezos-client gen keys MyBaker --encrypted'

This will create a key pair and encrypt it on disk using the password
you provided.

\subsubsection{Using Ledger Nano S}

Using a Ledger Nano S provides additional security since the private
key will never leave the device. Here is a list of steps you need to
perform in order to have a working setup:

\begin{footnotesize}
\begin{enumerate}
\item Initialize your Ledger with the 24 mnemonic words
\item Upgrade to latest firmware \& install Tezos Baking app from
  LedgerLive
\item Connect the Ledger to your baking server
\end{enumerate}
\end{footnotesize}

To get a listing of Ledger related commands handled by
\texttt{tezos-client}:

\lstinline[style=BashInputStyle]'$ tezos-client man ledger'

In order to list connect Ledgers which are detected by the software,
do:

\lstinline[style=BashInputStyle]'$ tezos-client list connected ledgers'

Each Ledger will be identified by mnemonic like
\texttt{hidden-tiger-crouching-dragon} in order to help you
distinguish between different Ledgers you might be connecting.

In order to add Ledger keys in your wallet, use a command like

\lstinline[style=BashInputStyle]'$ tezos-client import secret key MyBaker <ledger_id>'

where \texttt{<ledger\_id>} is a string like {\footnotesize \texttt{ledger://hidden-tiger-crouching-dragon/ed25519/0'/0'}}

At this point you will have a new key added to your wallet called
\texttt{MyBaker}.

The next step is to setup the Ledger for baking with this key. Ledgers
have a protection against double baking. You first need to
\emph{setup} your Ledger to authorize one of its key to bake. You can
optionally setup the \emph{high watermark}, i.e. the level under which
the Ledger will refuse to sign a block or endorsement in order to
avoid double baking.

\lstinline[style=BashInputStyle]'$ tezos-client setup ledger to bake for MyBaker'

At this point your Ledger should be ready to sign blocks for your
daemons.

\subsection{Register as a delegate}

In order to bake, two things are required

\begin{footnotesize}
\begin{enumerate}
\item A balance of at least one roll, i.e. XTZ 8000\footnote{On
  \today. In the past, a roll was XTZ 10000}
\item Be registered as a delegate at least 7 cycles before the current cycle
\end{enumerate}
\end{footnotesize}

To register as a delegate, use:

\lstinline[style=BashInputStyle]'$ tezos-client register key MyBaker as delegate'

You then need to wait for 7 cycles to have baking and endorsing slots
allowed to your baker. It means that if the current cycle is $n$, you
might start baking at cycle $n+7$ or $n+8$ depending on whether or not
you registered your delegate with a balance more than one or equal to
one roll before cycle $n$'s snapshot\footnote{See
  \url{http://tezos.gitlab.io/mainnet/whitedoc/proof_of_stake.html}}.

\section{Configure and Run}

\subsection{Tezos Node}

Baking daemons are clients to the Tezos node. Therefore you need to
run (at least!) one node in order to be able to bake. As explained
earlier, the recommanded operation mode for a baking mode is the
\textbf{full} mode. The \textbf{rolling} mode is not suitable for
baking.

Detailed instructions on how to run the node is outside the scope of
this tutorial. We mention here import relevant configuration items for
a node used for baking.

\subsubsection{Trusted peers}

Since the node will broadcast blocks and endorsements you created to
the network, you probably want to avoid giving your IP address
information to untrusted peers as this could provide a cue to guess
your physical location. In order to activate this feature, you need to
enable the \texttt{--private-mode} to connect only to a fixed set of
peers and optionally use \texttt{--no-bootstrap-peers} in conjunction
with \texttt{--peers} arguments to only connect to peers you
trust. This can be done either by providing command-line arguments
when you launch the node, or in the node configuration file.

Most bakers use a layered network architecture where they have a swarm
of \emph{front nodes} connecting to the rest of the network and a
baking node only connected to those front nodes, often with a
dedicated link that makes it much harder to infer the location of the
baking node from front nodes. This way, the rest of the network only
see your front nodes baking blocks and endorsements whereas the real
baking is done in the hidden baking node.

\subsubsection{Local-only RPC port}

It is mandatory that the rest of the internet do not have access to
the RPC port\footnote{Usually TCP 8732} of the node. Indeed, some RPC
commands are particularly demanding in terms of performance and it
would thus be trivial for an attacker to successfully DDoS your node by
repeatedly requesting such commands.

\subsection{Baking daemons}

\subsubsection{Run}

The \texttt{tezos-baker-*} daemon needs to have local access to a node
working directory in order to function, meaning you need to run it in
the same machine than the node. Other daemons can connect to a remote
node, however it is still recommended to run them close\footnote{On
  the same network or host} to a node for performance reasons.\\

\lstinline[style=BashInputStyle]'$ tezos-baker-004-Pt24m4xi run with local node ~/.tezos-node'

\lstinline[style=BashInputStyle]'$ tezos-endorser-004-Pt24m4xi run'

\lstinline[style=BashInputStyle]'$ tezos-accuser-004-Pt24m4xi run'

\subsubsection{Handling protocol changes}

Tezos is a self-amending ledger, meaning that bakers vote for new
protocol upgrades that can change most of Tezos operation, including
the consensus algorithm and the way blocks are baked. For this reason,
the baker, endorser and accuser daemons are tied to a specific version
of a protocol. Most of protocol upgrades so far did not change
fundamentally the way the consensus is achieved, but nonetheless, in
order to correctly handle protocol upgrades, you need to run one set
of daemons per protocol. The current set of daemons will operate until
the last block of the current protocol, and then stop, whereas the
“next” set of daemons will start operating on the first block of the
new protocol.

Running both sets of daemons ensure your setup will correctly
transition from one protocol to another and that you will not miss
baking blocks or endorsements.

The current protocol is \texttt{004-Pt24m4xi}. In the following we
show how to run daemons with this specific protocol version as an
example.

%% \tableofcontents

\end{document}
