\documentclass[a4paper,twocolumn,10pt]{article}
\usepackage{xcolor}
\usepackage{listings}
\usepackage{hyperref}

\lstdefinestyle{BashInputStyle}{
  language=bash,
  basicstyle=\small\sffamily,
  numbers=left,
  numberstyle=\tiny,
  numbersep=3pt,
  frame=tb,
  columns=fullflexible,
  backgroundcolor=\color{yellow!20},
  linewidth=0.9\linewidth,
  xleftmargin=0.1\linewidth
}

\title{Tezos Baking Cheat Sheet}
\date{\today}
\author{Vincent Bernardoff}

\begin{document}
\twocolumn
\maketitle
\tableofcontents
\section{Introduction}

This document intends to be a handy cheat sheet for Tezos
operators. It details the installation and operation of Tezos
software. It is a complement to the original
documentation\footnote{\url{http://tezos.gitlab.io}} already
available. It summarizes the different steps needed to compile and run
a \emph{node} and associated \emph{baker}, \emph{endorser}, and optionally
\emph{accuser} daemons which are necessary to participate in the PoS
consensus process.

The emphasized terms (in \emph{italic}) or quoted is technical
vocabulary that is used in Tezos source code and thus often commonly
used to talk about Tezos concepts. We encourage you to use it when you
discuss Tezos things with your peers since it is more precise and will
better describe what you are talking about.

Tezos uses LPoS (\emph{liquid proof of stake}) to achive
consensus. The proof of stake algorithm\footnote{Called \emph{Emmy} in
  reference to Emmy Noether} implemented in Tezos nodes attributes
block creation rights (“slots”) to stakeholders at random,
proportional to the size of their stake.

Operators (“bakers”) will need to initialize themselves as
\emph{delegates}, then the proof of stake algorithm run by nodes will,
after some time, attribute them block creation (“baking”) rights to
them.

The proof of stake algorithm also comprise an additional mechanism
which increase
security\footnote{\url{https://blog.nomadic-labs.com/analysis-of-emmy.html}}:
32 \emph{endorsing} rights per block are attributed proportionally to
bakers. Like baking blocks, endorsing blocks is rewarded by the
network hence the bakers are incentivized to do it.

Finally, bakers can also run an \emph{accuser} daemon, which watches
the chain and try to detect reprehensible behaviour of other network
participants. Upon finding such transgressions, the daemon can inject
a \emph{denunciation} operation in the chain, which is also rewarded.

It is your responsability as an operator (“baker”) to properly
register as a \emph{delegate}, then launch and ensure proper operation
of at least the \emph{node}, the \emph{baker} and \emph{endorser}
daemons, and optionally the \emph{accuser}.

The following is a brief description of the different components
software and the interaction between those components. Configuration
details and precise instructions on how to run it will be explained in
a later section.

\subsection{\texttt{tezos-node}}

The Tezos node is the cornerstone of the network, the P2P node which
interract (“gossip”) with the rest of the network and accepts clients
commands (RPCs). It implements \emph{Emmy}, the consensus algorithm,
the \emph{Michelson} virtual machine, and so on. It accepts messages
from its P2P peers in a dedicated (configurable) TCP port and from
this information, will decides which messages it considers genuine and
will build a blockchain from them. It also listens on client requests
on another TCP port (RPC). All other executables provided with the
node are clients, which role is detailed below.

\subsection{\texttt{tezos-client}}

\emph{tezos-client} is a tool for querying the node and creating and
injecting (\emph{injecting}) \emph{operations}, such as transactions,
smart contract origination, voting, and so on. It is not directly
related to baking but is necessary to create the cryptographic keys
used by baking, endorsing and accusing daemons.

\subsection{\texttt{tezos-baker-*}}

The role of \texttt{tezos-baker-*} is to automatically create block
and injection them in the node, which will broadcast them to its
peers. Its operation is as follows:

\begin{footnotesize}
\begin{enumerate}
\item Baker has access to a wallet with delegates
\item Node notifies the baker of new blocks on the chain
\item Baker checks with the shell if its delegates have slots
\item Baker sleeps until its time to bake, then wakes up and…
\item Queries the shell’s current mempool
\item Build a block with operations
\item Inject blocks to the node
\end{enumerate}
\end{footnotesize}

\subsection{\texttt{tezos-endorser-*}}

The role of \texttt{tezos-endorser-*} is to automatically inject
endorsement operations in the node. Its operation is as follows:

\begin{footnotesize}
\begin{enumerate}
\item Endorser has access to a wallet with delegates
\item Shell notifies the endorser of new blocks on the chain
\item Endorser checks with the shell if its delegates have endorsements slots
\item Endorse sleeps until its time to endorse, then wakes up and…
\item Prepare an endorsement operation
\item Inject the endorsement to the node
\end{enumerate}
\end{footnotesize}

\subsection{\texttt{tezos-accuser-*}}

The role of \texttt{tezos-accuser-*} is to automatically inject proof
of other bakers' misbehaviour in the node. Its operation is as
follows:

\begin{footnotesize}
\begin{enumerate}
\item Inspect the injected blocks and endorsement (via the shell)
\item When a delegate behaves in a malicious way:
  \begin{itemize}
    \item A delegate injected two blocks at a same level on two different branches
    \item A delegate endorsed two different blocks at a same level
  \end{itemize}
  → Inject a denounciation operation
\item The incriminated user will be slashed of all its rewards and bond deposits
\end{enumerate}
\end{footnotesize}

\subsection{\texttt{tezos-signer}}

\texttt{tezos-signer} is an optional mechanism useful to provide
additional security in some configurations. The aforementioned daemons
need to access a wallet in order to function. This wallet can either
have local keys (encrypted on disk, or on HSM\footnote{Ledger Nano S
  is supported by default.}). In some situation you would like to put
those keys in a different machine than the one the daemons are running
on. In this situation, you can add \emph{remote keys} to the
client. When used to sign data, the data to be signed will be sent to
a remote server who handles the signing and returns the
result. \texttt{tezos-signer} is such a server compatible with the
rest of Tezos software for remote signing.

\section{Prerequisite}
Detailed\footnote{\url{http://tezos.gitlab.io/mainnet/introduction/howtoget.html}}
instructions of how to obtain Tezos software can be found of the main
Tezos documentation, and are outside the scope of this document. There
are two means, either build the software directly from source code
(needs a working OCaml installation with OPAM), or using a Docker
image. Use whichever you feel most confortable with. OCaml programmers
will probably choose the former, while devops will opt for the latter.

It is \textbf{mandatory} to use NTP or a similar mechanism to keep
your server on time when runnning Tezos deamons. Indeed, the consensus
algorithm is synchronous\footnote{As Tezos evolves, this can change in
  the future}, and therefore requires correct system time to function.

Using SSDs is also strongly recommended. Tezos software might improve
its performance in the future and work fine on legacy hard disks.

%% \lstinline[style=BashInputStyle]´$ apt-get --purge remove rubygems´

\section{Prepare}
\subsection{Key management}
\section{Run}
\subsection{Tezos Node}
\subsection{Tezos Baker}
\subsection{Tezos Endorser}
\subsection{Tezos Accuser}
\subsection{Protocol change}
\end{document}
